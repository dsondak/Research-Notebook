%----------------------------------------------------------------------------------------
%	DEFINITION OF EXPERIMENTS
%----------------------------------------------------------------------------------------

% Template: \newexperiment{<abbrev>}[<short form>]{<long form>}
% <abbrev> is the reference to use later in the .tex file in \experiment{}, the <short form> is only used in the table of contents and running title - it is optional, <long form> is what is printed in the lab book itself

\newexperiment{ChemFEniCS}[Chemistry in FEniCS]{Implementation of Chemical Reaction Terms in FEniCS}
\newsubexperiment{DataTypes}[Data Types in FEniCS]{Be careful with data types in FEniCS}
\newsubexperiment{AltImplement}[Alt. C++ Interface]{Alternate implementation of C++ interface}
\newsubexperiment{Jacobians}[Manual Jacobians]{Manual calculation of reaction rate Jacobian}

\newexperiment{StochasticOP}[Stochastic Operator]{Development of the Stochastic Operator for Model Inadequacy}
\newsubexperiment{Cathcalls}[Catchall Reactions]{Development of the Catchall Reactions}
\newsubexperiment{Input File}[Input File]{Input File}

\newexperiment{0D Reactor}[0D reactor]{Development of the zero-D reactor software}
\newsubexperiment{Heating Rate}[Heatin Rate]{Implementation of Heating Rate}
\newsubexperiment{Reduced Model Calibration}[Calibration]{Calibration of the Reduced Model}

\newexperiment{VMS-ThermalConv}[Drekar VMS TC]{Implementation of VMS terms for thermal convection.}
\newsubexperiment{Building Drekar}[Building Drekar]{Notes on building Drekar}

\newexperiment{Janus Particles}[Janus Particles]{Getting results for Janus particles project}

