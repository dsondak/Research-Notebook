%----------------------------------------------------------------------------------------
%	LAB BOOK CONTENTS
%----------------------------------------------------------------------------------------

\labday{Thursday, 18 August, 2016}

%-----------------------------------------

\experiment{0D Reactor}
A few notes on the energy equation for the 0D reactor.  After ignoring all spatial
dependence the energy equation becomes,
\begin{align}
  \rho C_{p}\odeone{T}{t} = \dot{\omega}_{T}^{\prime} + \dot{\mathcal{Q}}.
\end{align}
Substituting for the heat release due to combustion we have
\begin{align}
  \rho C_{p}\odeone{T}{t} = -\sum_{k=1}^{N}h_{k}\lr{T}\dot{\omega}_{k} + \dot{\mathcal{Q}}.
\end{align}
Note that all quantities are on a mass basis.  The 0D reactor uses molar quantities.
\texttt{Antioch} returns mass quantities.  We need to check the units to make sure 
everything works out.

The species equation becomes
\begin{align}
  \odeone{\rho Y_{k}}{t} = \dot{\omega}_{k}.
\end{align}
Note that
\begin{align}
  \left[X_{k}\right] = \rho \frac{Y_{k}}{W_{k}} \Rightarrow \rho Y_{K} = W_{K}\left[X_{k}\right].
\end{align}
Now we have
\begin{align}
  \dot{x}_{k} = \odeone{\left[X_{k}\right]}{t} = \frac{\dot{\omega}_{k}}{W_{k}}.
\end{align}
And so the energy equation becomes
\begin{align}
  \rho C_{p}\odeone{T}{t} = - \sum_{k=1}^{N}h_{k}\lr{T}W_{k}\dot{x}_{k} + \dot{\mathcal{Q}}.
\end{align}
The specific heat is
\begin{align}
  C_{p} &= \sum_{k=1}^{N}C_{pk}Y_{k} \\
        &= \sum_{k=1}^{N}C_{pk}\frac{W_{k}}{\rho}x_{k}.
\end{align}
Then
\begin{align}
  \odeone{T}{t} = \frac{\displaystyle -\sum_{k=1}^{N}h_{k}\lr{T}W_{k}\dot{x}_{k} + \dot{\mathcal{Q}}}{\displaystyle \sum_{k=1}^{N}C_{pk}W_{k}x_{k}}.
\end{align}

%::::::::::::::::::::::::::::::::::::::::::::::::::::::::
\labday{Friday, 19 August, 2016}
%::::::::::::::::::::::::::::::::::::::::::::::::::::::::

\experiment{StochasticOP}
Trying to calibrate the stochastic operator for chemical kinetics.
Having the same problems as before with the reduced model.  Namely,
the rejection rates are simply too high and the log likelihoods are
too small.

I'm trying a few things:
\begin{itemize}
  \item Turn off stochastic operator to try to recover calibration from
        reduced model.
  \item Modify global temperature dependence to try to get better
        behavior.
  \item Played a bit with the algorithm parameters (adaptation interval
        length, etc).
\end{itemize}

Waiting to see the chains.  The global temperature dependence enters the
operator via a prefactor as
\begin{align}
  \odeone{\mathbf{x}}{t} = \mathbf{f}\lr{\mathbf{x}} + 
      g\lr{T}\left[\mathcal{S}\mathbf{x} + \mathcal{A}\lr{\mathbf{x}}\right].
\end{align}
The form that I have selected for the prefactor is
\begin{align}
  g\lr{T} = \frac{1}{2}\exp\lr{-\frac{T_{ag}}{T}}
    \left[\tanh\lr{\frac{T - T_{gi}}{\delta}} - \tanh\lr{\frac{T - T_{adi}}{\delta}}\right].
\end{align}
This introduces three new parameters to calibrate:
\begin{enumerate}
  \item $T_{ag}$, the global activation energy;
  \item $T_{gi}$, global ignition temperature;
  \item $T_{adi}$, global adiabatic temperature.
\end{enumerate}
\begin{figure}[ht!]
  \includegraphics[width=\textwidth]{August/global_T.pdf}
  \caption{Prefactor for the stochastic operator.}
  \label{fig:gofT}
\end{figure}
Figure~\ref{fig:gofT} shows the behavior of the prefactor.  The hope is that this
allows the stochastic operator to turn on and off as needed.  The Arrhenius form
basically just sets the amplitude of the prefactor and the premultiplication of 
$1/2$ just scales the hyperbolic tangents to unity (no real reason to do this
other than it helps humans see what's going on).  The smoothing factor
$\delta$ is selected to be unity.
