%----------------------------------------------------------------------------------------
%	LAB BOOK CONTENTS
%----------------------------------------------------------------------------------------

%::::::::::::::::::::::::::::::::::::::::::::::::::::::::
\labday{Friday, 2 September, 2016}
%::::::::::::::::::::::::::::::::::::::::::::::::::::::::

\experiment{StochasticOP}
We have decided to only calibrate the inadequacy parameters (and hyperparameters)
and to freeze the model parameters.  The reason behind this is because inference
of the model parameters is particularly difficult as evidenced by the high
rejection rates.  Our current approach is defensible in the near-term because we
are really only interested in calibration of the inadequacy parameters.  We are
accepting the model parameters as truth (and truth be told, they are pretty 
darn good).

I have calibrated three versions of the inadequacy model: 1.) Using no global
temperature dependence; 2.) using a global temperature dependence in Arrhenius
form and 3.) using a global temerature dependence in the hyperbolic tangent form.
With no global temperature dependence (and no model parameter inference) the 
calibration is remarkably quick for the first $30000$ iterations.  However, 
the reaction rates for the catchall reactions eventually cause the system to 
become quite stiff and at iteration $\sim 34000$ the model evaluation times 
are on the order of tens of minutes.  The rejection rates are still rather ``low''
by our standards.  They hover somewhere between $76\%$ and $85\%$.  However,
the likelihoods are very small (essentially zero).  For example, the loglikelihood
at iteration $36000$ is around $-25000$.

Using the Arrehnius global temperature dependence, the model evaluations are
a bit slower and the rejecton rates are considerably higher ($\sim 96\%$).  
The likelihoods are larger but still essentially zero ($-22000$ at iteration
$36000$).

With the global hyperbolic temperature dependence, the model evaluations 
are comparable to those from the global Arrhenius temperature dependence. 
The rejection rates are $\sim 91\%$ and the likelihoods are larger still 
(but still essentially zero, e.g. $-7500$ at iteration $38000$).

Note that the calibrations described above were run using data from 
ten calibration points as opposed to data from twenty points as I had been
doing previously.  The main point is that the calibration data should be 
a reasonable representation of curves.  If ten points gives a decent 
representation then adding more points causes more work and does not 
give much in return.

Additionally, the log likelihood actually has an additional factor that 
I am not reporting.  Recall, the likelihood that we are using is 
\begin{align}
  L = \frac{1}{\lr{2\pi}^{n_{d}/2}\left|\Sigma\right|^{1/2}}
     \exp\lr{-\frac{1}{2}\lr{\mathbf{d} - \mathbf{x}}^{T}\Sigma^{-1}\lr{\mathbf{d} - \mathbf{x}}}
\end{align}
where $\Sigma$ is a diagonal matrix of variances, $n_{d}$ is the number of data points (here
$n_{d} = 10$, $\mathbf{d}$ is the truth data and $\mathbf{x}$ is the model data.  Note that for a 
diagonal matrix 
\begin{align}
  \left|\Sigma\right| = \prod_{i=1}^{n}\sigma_{i}
\end{align}
where $n$ is the size of the matrix.  Here $\sigma_{i} = 5\cdot 10^{-3}$ for $i=1,\ldots 7$
(for the first seven species), $\sigma_{8} = 2\cdot 10^{3}$ for the temperature 
and $\sigma_{9} = 10$ for the ignition temperature data.  Hence
\begin{align}
  \left|\Sigma\right| = \lr{5\cdot 10^{-3}}^{7} \cdot 2\times 10^{3} \cdot 10.
\end{align}
The log-likelihood is 
\begin{align}
  \log\lr{L} = -\log\lr{\lr{2\pi}^{n_{d}/2}\left|\Sigma\right|^{1/2}} - 
      \frac{1}{2}\lr{\mathbf{d} - \mathbf{x}}^{T}\Sigma^{-1}\lr{\mathbf{d} - \mathbf{x}}.
\end{align}
With our values, the first term is $4.403$.

\experiment{Combustion Model Inadequacy}
Trying to propagate the fully stochastic operator model through the flamelet model to generate 
the uncertain flamelet.

\subexperiment{Diffusion Flame}
Considerable changes had to be made to the 
