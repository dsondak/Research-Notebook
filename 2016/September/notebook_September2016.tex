%----------------------------------------------------------------------------------------
%	LAB BOOK CONTENTS
%----------------------------------------------------------------------------------------

%::::::::::::::::::::::::::::::::::::::::::::::::::::::::
\labday{Friday, 2 September, 2016}
%::::::::::::::::::::::::::::::::::::::::::::::::::::::::

\experiment{StochasticOP}
We have decided to only calibrate the inadequacy parameters (and hyperparameters)
and to freeze the model parameters.  The reason behind this is because inference
of the model parameters is particularly difficult as evidenced by the high
rejection rates.  Our current approach is defensible in the near-term because we
are really only interested in calibration of the inadequacy parameters.  We are
accepting the model parameters as truth (and truth be told, they are pretty 
darn good).

I have calibrated three versions of the inadequacy model: 1.) Using no global
temperature dependence; 2.) using a global temperature dependence in Arrhenius
form and 3.) using a global temerature dependence in the hyperbolic tangent form.
With no global temperature dependence (and no model parameter inference) the 
calibration is remarkably quick for the first $30000$ iterations.  However, 
the reaction rates for the catchall reactions eventually cause the system to 
become quite stiff and at iteration $\sim 34000$ the model evaluation times 
are on the order of tens of minutes.  The rejection rates are still rather ``low''
by our standards.  They hover somewhere between $76\%$ and $85\%$.  However,
the likelihoods are very small (essentially zero).  For example, the loglikelihood
at iteration $36000$ is around $-25000$.

Using the Arrehnius global temperature dependence, the model evaluations are
a bit slower and the rejecton rates are considerably higher ($\sim 96\%$).  
The likelihoods are larger but still essentially zero ($-22000$ at iteration
$36000$).

With the global hyperbolic temperature dependence, the model evaluations 
are comparable to those from the global Arrhenius temperature dependence. 
The rejection rates are $\sim 91\%$ and the likelihoods are larger still 
(but still essentially zero, e.g. $-7500$ at iteration $38000$).

Note that the calibrations described above were run using data from 
ten calibration points as opposed to data from twenty points as I had been
doing previously.  The main point is that the calibration data should be 
a reasonable representation of curves.  If ten points gives a decent 
representation then adding more points causes more work and does not 
give much in return.

Additionally, the log likelihood actually has an additional factor that 
I am not reporting.  Recall, the likelihood that we are using is 
\begin{align}
  L = \frac{1}{\lr{2\pi}^{n_{d}/2}\left|\Sigma\right|^{1/2}}
     \exp\lr{-\frac{1}{2}\lr{\mathbf{d} - \mathbf{x}}^{T}\Sigma^{-1}\lr{\mathbf{d} - \mathbf{x}}}
\end{align}
where $\Sigma$ is a diagonal matrix of variances, $n_{d}$ is the number of data points (here
$n_{d} = 10$, $\mathbf{d}$ is the truth data and $\mathbf{x}$ is the model data.  Note that for a 
diagonal matrix 
\begin{align}
  \left|\Sigma\right| = \prod_{i=1}^{n}\sigma_{i}
\end{align}
where $n$ is the size of the matrix.  Here $\sigma_{i} = 5\cdot 10^{-3}$ for $i=1,\ldots 7$
(for the first seven species), $\sigma_{8} = 2\cdot 10^{3}$ for the temperature 
and $\sigma_{9} = 10$ for the ignition temperature data.  Hence
\begin{align}
  \left|\Sigma\right| = \lr{5\cdot 10^{-3}}^{7} \cdot 2\times 10^{3} \cdot 10.
\end{align}
The log-likelihood is 
\begin{align}
  \log\lr{L} = -\log\lr{\lr{2\pi}^{n_{d}/2}\left|\Sigma\right|^{1/2}} - 
      \frac{1}{2}\lr{\mathbf{d} - \mathbf{x}}^{T}\Sigma^{-1}\lr{\mathbf{d} - \mathbf{x}}.
\end{align}
With our values, the first term is $4.403$.  It has very little effect on the full
log likelihood.

\experiment{Combustion Model Inadequacy}
Trying to propagate the fully stochastic operator model through the flamelet model to generate 
the uncertain flamelet.

\subexperiment{Diffusion Flame}
Considerable changes had to be made to the Cantera input files to propagate the stochastic 
operator through the diffusion flame.  Rebecca's original template was incomplete.  First of 
all, the NASA polynomials of order $n$ must have $n$ coefficients specified.  Unspecified 
coefficients are not assumed to be zero and an error is thrown.  The NASA polynomial 
coefficients do not map directly to the quadratic form assumed by Rebecca for the inference. 
Recall that the NASA polynomial for the enthalpy is given by 
\begin{align}
  \frac{h}{RT} = a_{0} + \frac{a_{1}}{2}T + \frac{a_{2}}{3}T^{2} + \frac{a_{3}}{4}T^{3} + 
       \frac{a_{4}}{5}T^{4} + \frac{a_{5}}{T}
\end{align}
whereas Rebecca's parameterization for the catchall species looks like 
\begin{align}
  h^{\prime} = \alpha_{0} + \alpha_{1}T + \alpha_{2}T^{2}.
\end{align}
Thus, to insure consistency in the input file, we have the relationships,
\begin{align}
  a_{0} = \frac{\alpha_{1}}{R}, \quad a_{1} = \frac{2\alpha_{2}}{R}, \quad a_{5} = \frac{\alpha_{0}}{R}
\end{align}
and $a_{i} = 0$ for $i=2, 3, 4$.

Another difficulty is how exactly to specify the catchall species.  We know that $\ce{H}^{\prime}$ 
is made up of hydrogen.  However, Cantera gets confused between two species which has the exact 
same compositions.  We initially tried to define our own atom (a dummy atom) which would have
the exact same properties as $\ce{H}$.  However, this resulted in Cantera errors.  Both
approaches caused Cantera to fail in the \texttt{set\_initial\_guess} method at the \texttt{equilibrate} 
step.  In that step, Cantera attempts to set a reasonable initial guess to kick off the simulation 
based off an equilibrium solution. 

The way that I got around these convergence issues was to change the \texttt{set\_initial\_guess} 
method in addition to the actual initial guesses (this is actually independent somehow).  
The \texttt{set\_initial\_guess} method must be called one way or another.  It appears to 
set up important solution objects.  To get the inadequacy model to run, I used a previous 
solution from the reduced model.  I also modified the \texttt{set\_initial\_guess} method 
to accept a flag called \texttt{calc\_eq}.  When this flag is set to \texttt{False} the 
\texttt{equilibrate} routine is not called and no initial guess is returned.  The 
\texttt{set\_initial\_guess} method still sets up the necessary solution objects for the 
simulation.  In our application, when a previous solution is passed to the diffusion flame
module, the flag \texttt{calc\_eq} is automatically set to \texttt{False}.  After 
\texttt{set\_initial\_guess} is called, the solution fields are populated based on the 
previous solution and the catchall fields are set to zero.  The inadequacy model 
simulation is then able to run to completion modulo time integration convergence 
problems (see below). 

An additional modification that had to be made was to add a global Arrhenius 
prefactor to the inadequacy model.  This was not a problem.  The activiation 
energy field was specified in the reactions corresponding to the inadequacy 
model.  Note that the calibration was performed for the global activation 
temperature which is related to the global activation energy simply through 
the universal gas constant via $T_{ag} = E_{ag} / R$. 

The time integration failed to converge in the absence of the global 
temperature dependence.  Using the global temperature dependence, the time 
integration was successful but I still don't know why it didn't converge 
originally.  Remember, the 0D reactor time integrator converges even in 
the absence of the global temperature dependence (although it does struggle 
once in a while).

In any case, we can look at solutions from the diffusion flame and compare 
the detailed, reduced and inadequacy models.  Note that the chains have 
still not been completely propagated through the diffusion flame so there 
may be some parameter values that are really tough to deal with.


































