%----------------------------------------------------------------------------------------
%	LAB BOOK CONTENTS
%----------------------------------------------------------------------------------------

\labday{Friday, 1 April, 2016}
\begin{enumerate}
  \item Tried to find reasonable value for the heating rate in the energy equation for the zero-D reactor.
  \item Finally got Drekar configured and compiled.  All tests, including extended tests, pass.  Errors when adding options for SUPG Energy evaluator.  Not clear what the errors are.
\end{enumerate}

%-----------------------------------------
\experiment{0D Reactor}
To do:
\begin{enumerate}
  \item Clean up code.  Add comments and document how it works.
  \item Find reasonable heating rate values.
  \item Run heating rate calibration.
\end{enumerate}
Today I just worked on the heating rate values.
%-----------------------------------------
\subexperiment{Heating Rate}
Recall that the energy equation is
\begin{align}
  \odeone{T}{t} = -\dfrac{\displaystyle\sum_{k=1}^{N}{u_{k}\lr{T}\odeone{x_{k}}{t}}}{\displaystyle\sum_{i=k}^{N}{c_{vk}x_{k}}} + Q_{T}
\end{align}
where $T$ is the temperature, $u_{k}\lr{T}$ is the internal energy of species $k$, $x_{k}$ is the molar concentration of species $k$ and $c_{vk}$ is the specific heat at constant volume of species $k$.  We have also included the heating rate $Q_{T}$ which may be a function of time or temperature.

\begin{tcolorbox}[colback=blue!5, colframe=blue!40!black, title=Estimate of Heating Rate]
  \begin{align*}
    Q_{T} \approx \odeone{T}{x}s_{L}
  \end{align*}
  where $s_{L}$ is the laminar flame speed.
\end{tcolorbox}

The laminar flame speed for hydrogen-air premixed flames at $\phi = 1$ is approximately $2$ m/s.  The energy equation in one dimension reduces to
\begin{align}
  \rho s_{L}\odeone{T}{x} = \odeone{}{x}\lr{\frac{\lambda}{c_{p}} \odeone{T}{x}}
\end{align}
where we assumed constant specific heat among species and no reactions.  Integrating this from $T_{u}$ to $T_{b}$ and solving for the temperature gradient gives
\begin{align}
  \odeone{T}{x} = \frac{\rho_{u}c_{p}s_{L}}{\lambda_{b}}\lr{T_{b}-T_{u}}.
\end{align}
Hence,
\begin{align}
  Q_{T} \approx \frac{\rho_{u}c_{p}s_{L}^{2}}{\lambda_{b}}\lr{T_{b}-T_{u}}.
\end{align}
If we now use some very rough values for air of $\rho_{u} = 1.225$, $c_{p} \approx 1200$ and $\lambda_{b}=0.05$ and we use $T_{b}\approx 3000$, $T_{u} = 1500$ with $s_{L}=4$ (all in SI units and temperatures in Kelvin) then $Q_{T}\approx 10^{8}$.  Note that $T_{b}$ and $T_{u}$ were taken from a simulation of the 0-D reactor without any source term.

%-----------------------------------------
\experiment{VMS-ThermalConv}
Worked on building Drekar.

\subexperiment{Building Drekar}

\labday{Week of April 4th, 2016}
\begin{itemize}
  \item Spent the week in Lausanne, Switzerland at SIAM UQ 16.  Made some notes about papers to read.  Will hopefully read some of them on the plane ride home.
  \item Thought about what plots to include in Janus particle paper but did not finish this.  Here's what I need to get:
    \begin{enumerate}
      \item Plot of $U\lr{t}$ for advection and no-advection cases with same values of $\alpha$ and $C$.
      \item Plot of logarithmic decay for no-advection case.
      \item Plots illustrating qualitative behavior of solution such as:  $U\lr{t}$ at for constant $C$ and various values of $\alpha$, $U\lr{t}$ for constant $\alpha$ and various values of $C$, illustration of different qualitative regimes such as peak velocity or not, plots of $t_{max}$ with $C$ and $\alpha$ when appropriate, relationship between final swimming velocity and $\alpha$ and $C$.
    \end{enumerate}
  \item Got motivated to use \fenics to do some tests of the old non-diagonal stabilization parameter.  First implement linear Burger's MHD (constant advection velocity and magnetic field coefficient), then do full Burger's MHD.  These tests will be \textit{steady} for now.  Then, when the solution looks decent, put in GLS stabilization and check solution again with the usual diagonal stabilization parameter.  Finally, put in the new non-diagonal stabilization parameter and compare the solutions.  If the solutions look good (which they should since we've done this before) then try out some two dimensional problems.  Will need to code up the full MHD equations.  Try to do Hartmann flow perhaps?  There may be some other interesting ones to try too.
\end{itemize}

%-----------------------------------------
%\end{addmargin}

\end{document}
