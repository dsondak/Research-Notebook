%%%%%%%%%%%%%%%%%%%%%%%%%%%%%%%%%%%%%%%%%
% Compact Laboratory Book
% LaTeX Template
% Version 1.0 (4/6/12)
%
% This template has been downloaded from:
% http://www.LaTeXTemplates.com
%
% Original author:
% Joan Queralt Gil (http://phobos.xtec.cat/jqueralt) using the labbook class by
% Frank Kuster (http://www.ctan.org/tex-archive/macros/latex/contrib/labbook/)
%
% License:
% CC BY-NC-SA 3.0 (http://creativecommons.org/licenses/by-nc-sa/3.0/)
%
% Important note:
% This template requires the labbook.cls file to be in the same directory as the
% .tex file. The labbook.cls file provides the necessary structure to create the
% lab book.
%
% The \lipsum[#] commands throughout this template generate dummy text
% to fill the template out. These commands should all be removed when 
% writing lab book content.
%
% HOW TO USE THIS TEMPLATE 
% Each day in the lab consists of three main things:
%
% 1. LABDAY: The first thing to put is the \labday{} command with a date in 
% curly brackets, this will make a new section showing that you are working
% on a new day.
%
% 2. EXPERIMENT/SUBEXPERIMENT: Next you need to specify what 
% experiment(s) and subexperiment(s) you are working on with a 
% \experiment{} and \subexperiment{} commands with the experiment 
% shorthand in the curly brackets. The experiment shorthand is defined in the 
% 'DEFINITION OF EXPERIMENTS' section below, this means you can 
% say \experiment{pcr} and the actual text written to the PDF will be what 
% you set the 'pcr' experiment to be. If the experiment is a one off, you can 
% just write it in the bracket without creating a shorthand. Note: if you don't 
% want to have an experiment, just leave this out and it won't be printed.
%
% 3. CONTENT: Following the experiment is the content, i.e. what progress 
% you made on the experiment that day.
%
%%%%%%%%%%%%%%%%%%%%%%%%%%%%%%%%%%%%%%%%%

%----------------------------------------------------------------------------------------
%	PACKAGES AND OTHER DOCUMENT CONFIGURATIONS
%----------------------------------------------------------------------------------------                               

\documentclass[fontsize=11pt, % Document font size
               paper=a4, % Document paper type
               oneside, % Shifts odd pages to the left for easier reading when printed, can be changed to oneside
               captions=tableheading,
               index=totoc,
               hyperref]{labbook}
 
\usepackage[bottom=10em]{geometry} % Reduces the whitespace at the bottom of the page so more text can fit

\usepackage[english]{babel} % English language
\usepackage{lipsum} % Used for inserting dummy 'Lorem ipsum' text into the template

\usepackage[utf8]{inputenc} % Uses the utf8 input encoding
\usepackage[T1]{fontenc} % Use 8-bit encoding that has 256 glyphs

\usepackage[osf]{mathpazo} % Palatino as the main font
\linespread{1.05}\selectfont % Palatino needs some extra spacing, here 5% extra
\usepackage[scaled=.88]{beramono} % Bera-Monospace
\usepackage[scaled=.86]{berasans} % Bera Sans-Serif

\usepackage{booktabs,array} % Packages for tables

\usepackage{amsmath} % For typesetting math
\usepackage{graphicx} % Required for including images
\usepackage{etoolbox}
\usepackage[norule]{footmisc} % Removes the horizontal rule from footnotes
\usepackage{lastpage} % Counts the number of pages of the document

\graphicspath{{/Users/dsondak/Documents/Research/Notebook/Research-Notebook/figures/2016/}{/h1/dsondak/Research-Notebook/figures/2016/}}

\usepackage{natbib}

\usepackage[dvipsnames]{xcolor}  % Allows the definition of hex colors
\definecolor{titleblue}{rgb}{0.16,0.24,0.64} % Custom color for the title on the title page
\definecolor{linkcolor}{rgb}{0,0,0.42} % Custom color for links - dark blue at the moment

\addtokomafont{title}{\Huge\color{titleblue}} % Titles in custom blue color
\addtokomafont{chapter}{\color{OliveGreen}} % Lab dates in olive green
\addtokomafont{section}{\color{Sepia}} % Sections in sepia
\addtokomafont{pagehead}{\normalfont\sffamily\color{gray}} % Header text in gray and sans serif
\addtokomafont{caption}{\footnotesize\itshape} % Small italic font size for captions
\addtokomafont{captionlabel}{\upshape\bfseries} % Bold for caption labels
\addtokomafont{descriptionlabel}{\rmfamily}
%\setcapwidth[r]{10cm} % Right align caption text
\setkomafont{footnote}{\sffamily} % Footnotes in sans serif

\deffootnote[4cm]{4cm}{1em}{\textsuperscript{\thefootnotemark}} % Indent footnotes to line up with text

\DeclareFixedFont{\textcap}{T1}{phv}{bx}{n}{1.5cm} % Font for main title: Helvetica 1.5 cm
\DeclareFixedFont{\textaut}{T1}{phv}{bx}{n}{0.8cm} % Font for author name: Helvetica 0.8 cm

\usepackage[nouppercase,headsepline]{scrpage2} % Provides headers and footers configuration
\pagestyle{scrheadings} % Print the headers and footers on all pages
\clearscrheadfoot % Clean old definitions if they exist

\automark[chapter]{chapter}
\ohead{\headmark} % Prints outer header

\setlength{\headheight}{25pt} % Makes the header take up a bit of extra space for aesthetics
\setheadsepline{.4pt} % Creates a thin rule under the header
\addtokomafont{headsepline}{\color{lightgray}} % Colors the rule under the header light gray

\ofoot[\normalfont\normalcolor{\thepage\ |\  \pageref{LastPage}}]{\normalfont\normalcolor{\thepage\ |\  \pageref{LastPage}}} % Creates an outer footer of: "current page | total pages"

% These lines make it so each new lab day directly follows the previous one i.e. does not start on a new page - comment them out to separate lab days on new pages
\makeatletter
\patchcmd{\addchap}{\if@openright\cleardoublepage\else\clearpage\fi}{\par}{}{}
\makeatother
\renewcommand*{\chapterpagestyle}{scrheadings}

% These lines make it so every figure and equation in the document is numbered consecutively rather than restarting at 1 for each lab day - comment them out to remove this behavior
\usepackage{chngcntr}
\counterwithout{figure}{labday}
\counterwithout{equation}{labday}

% For color boxes
\usepackage{tcolorbox}

% For chemistry
\usepackage{mhchem}

% Hyperlink configuration
\usepackage[
    pdfauthor={}, % Your name for the author field in the PDF
    pdftitle={Laboratory Journal}, % PDF title
    pdfsubject={}, % PDF subject
    bookmarksopen=true,
    linktocpage=true,
    urlcolor=linkcolor, % Color of URLs
    citecolor=linkcolor, % Color of citations
    linkcolor=linkcolor, % Color of links to other pages/figures
    backref=page,
    pdfpagelabels=true,
    plainpages=false,
    colorlinks=true, % Turn off all coloring by changing this to false
    bookmarks=true,
    pdfview=FitB]{hyperref}

\usepackage[stretch=10]{microtype} % Slightly tweak font spacing for aesthetics

%\setlength\parindent{0pt} % Uncomment to remove all indentation from paragraphs

%----------------------------------------------------------------------------------------
%       NEW COMMANDS
%----------------------------------------------------------------------------------------

\newcommand{\fenics}{FEniCS \ }
\newcommand{\lr}[1]{\left(#1\right)}
\newcommand{\pdeone}[2]{\frac{\partial #1}{\partial #2}}
\newcommand{\pden}[3]{\frac{\partial^{#3} #1}{\partial #2^{#3}}}
\newcommand{\odeone}[2]{\frac{\mathrm{d} #1}{\mathrm{d} #2}}
\newcommand{\nup}[1]{\nu_{#1}^{\prime\prime}}
\newcommand{\nur}[1]{\nu_{#1}^{\prime}}
\newcommand{\frate}[1]{K_{#1}^{\left(f\right)}}
\newcommand{\rrate}[1]{K_{#1}^{\left(r\right)}}

%----------------------------------------------------------------------------------------
%	DEFINITION OF EXPERIMENTS
%----------------------------------------------------------------------------------------

% Template: \newexperiment{<abbrev>}[<short form>]{<long form>}
% <abbrev> is the reference to use later in the .tex file in \experiment{}, the <short form> is only used in the table of contents and running title - it is optional, <long form> is what is printed in the lab book itself

\newexperiment{ChemFEniCS}[Chemistry in FEniCS]{Implementation of Chemical Reaction Terms in FEniCS}
\newsubexperiment{DataTypes}[Data Types in FEniCS]{Be careful with data types in FEniCS}
\newsubexperiment{AltImplement}[Alt. C++ Interface]{Alternate implementation of C++ interface}
\newsubexperiment{Jacobians}[Manual Jacobians]{Manual calculation of reaction rate Jacobian}

\newexperiment{StochasticOP}[Stochastic Operator]{Development of the Stochastic Operator for Model Inadequacy}
\newsubexperiment{Cathcalls}[Catchall Reactions]{Development of the Catchall Reactions}
\newsubexperiment{Input File}[Input File]{Input File}

\newexperiment{0D Reactor}[0D reactor]{Development of the zero-D reactor software}
\newsubexperiment{Heating Rate}[Heatin Rate]{Implementation of Heating Rate}
\newsubexperiment{Reduced Model Calibration}[Calibration]{Calibration of the Reduced Model}

\newexperiment{VMS-ThermalConv}[Drekar VMS TC]{Implementation of VMS terms for thermal convection.}
\newsubexperiment{Building Drekar}[Building Drekar]{Notes on building Drekar}

\newexperiment{Janus Particles}[Janus Particles]{Getting results for Janus particles project}



%----------------------------------------------------------------------------------------

\begin{document}

%----------------------------------------------------------------------------------------
%	TITLE PAGE
%----------------------------------------------------------------------------------------

\title{\textcap{Laboratory Journal \\[1cm]  
%\textaut{Beginning 23-03-2016}
}
}

\author{
    \textaut{David Sondak}\\ \\ % Your name
}
\date{} % No date by default, add \today if you wish to include the publication date

\maketitle % Title page

\printindex
\tableofcontents % Table of contents
\newpage % Start lab look on a new page

%\begin{addmargin}[4cm]{0cm} % Makes the text width much shorter for a compact look

\pagestyle{scrheadings} % Begin using headers

%----------------------------------------------------------------------------------------
%	LAB BOOK CONTENTS
%----------------------------------------------------------------------------------------

%\input{2016/March/notebook_March2016.tex}
%%----------------------------------------------------------------------------------------
%	LAB BOOK CONTENTS
%----------------------------------------------------------------------------------------

\labday{Friday, 1 April, 2016}
\begin{enumerate}
  \item Tried to find reasonable value for the heating rate in the energy equation for the zero-D reactor.
  \item Finally got Drekar configured and compiled.  All tests, including extended tests, pass.  Errors when adding options for SUPG Energy evaluator.  Not clear what the errors are.
\end{enumerate}

%-----------------------------------------
\experiment{0D Reactor}
To do:
\begin{enumerate}
  \item Clean up code.  Add comments and document how it works.
  \item Find reasonable heating rate values.
  \item Run heating rate calibration.
\end{enumerate}
Today I just worked on the heating rate values.
%-----------------------------------------
\subexperiment{Heating Rate}
Recall that the energy equation is
\begin{align}
  \odeone{T}{t} = -\dfrac{\displaystyle\sum_{k=1}^{N}{u_{k}\lr{T}\odeone{x_{k}}{t}}}{\displaystyle\sum_{i=1}^{N}{c_{vk}x_{k}}} + Q_{T}
\end{align}
where $T$ is the temperature, $u_{k}\lr{T}$ is the internal energy of species $k$, $x_{k}$ is the molar concentration of species $k$ and $c_{vk}$ is the specific heat at constant volume of species $k$.  We have also included the heating rate $Q_{T}$ which may be a function of time or temperature.

\begin{tcolorbox}[colback=blue!5, colframe=blue!40!black, title=Configuring Drekar with Extended System Tests]
  x
\end{tcolorbox}

%-----------------------------------------
\experiment{VMS-ThermalConv}
Worked on building Drekar.

\subexperiment{Building Drekar}

%-----------------------------------------
%\end{addmargin}

\end{document}

%%----------------------------------------------------------------------------------------
%	LAB BOOK CONTENTS
%----------------------------------------------------------------------------------------

\labday{Friday, 13 May, 2016}


%-----------------------------------------

\experiment{0D Reactor}

\subexperiment{Reduced Model Calibration}
Working on getting chains for the reduced model.  The problem is that the chains are taking a while to settle down and that the rejection rate is high ($\sim 95\% - 99\%$).  We can deal with this in two ways.  First, to deal with the rejection rate, we can change algorithmic parameters such as the initial chain length, the adaptivity interval, scaling of the adaptivity and etc.  It can also be beneficial to change the proposal covariance matrix, $V$.  Making it smaller helps reduce the rejection rates but requires more time to explore the state space.  Here are the changes that I've made so far to Rebecca's code with regards to this issue:
\begin{itemize}
  \item $V$ was originally $V = \text{diag}\lr{10^{-4}}$.  I've changed it to $V = \text{diag}\lr{10^{-6}}$.
  \item The prior covariance matrix, $V_{0}$ was originally $V_{0} = \text{diag}\lr{10^{-4}}$.  This basically corresponded to an overly informative prior and therefore it was not clear what the inference problem was even learning.  I've changed this to be $V_{0} = \text{diag}\lr{\sigma_{i}^{2}}$ where $\sigma_{i} = \mu_{i}/10$ and $\mu_{i}$ is the mean of the $\text{i}^{\text{th}}$ parameter.  Hence the covariance is a diagonal matrix wherein the standard deviation is $10\%$ of the mean.
  \item I changed the variance on the observations from $10^{-4}$ to $10^{-3}$ for the concentrations.  The temperature variance was originally $1.0$ and I have now set it to be $10^{3}$ which is basically corresponds to a standard deviation that is $1.7\%$ of the temperature difference between post- and pre-ignition.  $3\%$ would be around $3\times 10^{3}$.
  \item Another change that I made was to make the likelihood for the temperature include additive Gaussian noise on the temperature observations.  Rebecca was using a likelihood that corresponded to an unknown type of observation error.
  \item A final change was to change the initial parameter values.  I ran a long change $10^{5}$ and noticed that the stationary values rather different than the values in the literature.  I eyeballed the final values from the inference (which was still not fully converged) and used those as the initial parameters.  The DRAM rejection rate is now around $70\%$ which is close to the rule of thumb that Damon gave me to me ($\approx 28\%$ acceptance rate).
\end{itemize} 

The other way to deal with our issue is to perform an optimization problem before starting the inference problem.  The optimization problem will attempt to find the MAP (Maximum a Posteriori) point and use this point as a seed for the MCMC (Markov Chain Monte Carlo) algorithm.  This will hopefully help to cut down on the burnin period.  The MAP point is the maximum of the posterior distribution.  Without knowing the MAP point, the MCMC algorithm will just start from whichever initial parameter values we provide it and search the state space until it begins to converge to the stationary values.  Once it finally reaches the stationary values then the posterior is being sampled.  It can take a very long time to reach the stationary point for arbitrary initial parameters.  This time is called burnin.  Now, if provide the initial parameters as an initial guess to an optimization algorithm then the optimizer will try to find the maximum of the posterior.  Once it finds the maximum, we can use those points as the starting point to the MCMC algorithm and therefore try to avoid the whole burnin issue and sample the posterior from the beginning.  There were some difficulties with this.
\begin{itemize}
  \item Had to figure out what the right options for the input file were.  Damon helped enormously with this.  He had to check in the lastest optimizer changes to \texttt{Queso} for me to fiddle with the optimizer parameters.
  \item I spent a long time trying to figure out the correct combination of parameters to the optimizer.  The problem was that the prior covariance spanned somewhere around $10$ to $11$ orders of magnitude.  This made the optimization problem outrageously difficult to solve.  As such, none of the optimizer algorithms gave anything.
  \item The way we figured the problem out was to set the likelihood to $1$.  This ensured that the optimizer was simply trying to find the maximum of the prior.  Also, to speed things up, I removed the model evaluations from the code.
  \item Since our variances are based off of the initial states, I decided to scale the initial states.  In particular, I divided the intial states by $100$.  This way, the maximum covariance is around $10^{-2}$ and the minimum is around $10^{-4}$.  Without the model, the optimizer converges rapidly.  With the model, things are very slow.  Still waiting to see if I get any results.
\end{itemize}

\end{document}

%%----------------------------------------------------------------------------------------
%	LAB BOOK CONTENTS
%----------------------------------------------------------------------------------------

\labday{Friday, 08 July, 2016}


%-----------------------------------------

I have been remiss in writing in the research journal.  Many things have happened.
I won't even bother trying to summarize.  Hopefully I can start to become regular
again.

\experiment{Nondiagonal Stabilization}
Digging up this old problem.  I never published it, but a good portion of the work is ready to go.
It seems that a nondiagonal stabilization parameter could be useful for MHD problems.  Assad and I
developed such a paramter for MHD and got some promising results.  I no longer have access to those
results, but I'd like to publish them.  I'm trying to use \fenics to generate the results.

At the moment, I have a Burger's implementation that solves the steady Burger's equation,
\begin{align}
 u\pdeone{u}{x} = \nu\pden{u}{x}{2}, \quad u\lr{x_{L}} = u_{L}, \ u\lr{x_{R}} = u_{R} \label{eq:burgers}
\end{align}
where the left ($u_{L}$) and right ($u_{R}$) boundary conditions are determined from
\begin{align}
  \frac{1}{2}\lr{x_{R}\lr{1-x_{*}} + x_{L}\lr{1+x_{*}}} = u_{*}
\end{align}
where $*$ can be either $L$ or $R$ indicating left or right.  Right now I am using
$x_{L} = -1$ and $x_{R} = 1$ which gives $u_{L} = -1$ and $u_{R} = 1$.  This gives
a ``shock'' at the center of the domain as shown in figure~\ref{fig:burgers_shock}.
\begin{figure}[h!]
  \centering
  \includegraphics[width=0.7\textwidth]{/h1/dsondak/Research-Notebook/figures/2016/July/burgers_shock.pdf}
  \caption{Illustration of \fenics solution to the steady Burger's equation~\eqref{eq:burgers} using $5000$ finite elements.}
  \label{fig:burgers_shock}
\end{figure}

Next, I need to figure out how to introduce the stabilization parameter into
\fenics.  I'll base this off of Umberto's \fenics implementation of the
stabilization parameter.  Once the stabilization parameter is working, I can 
code up the magnetic field equation, followed by the nondiagonal stabilization
parameter.  Then I will introduce the nondiagonal stabilization parameter and
start compiling the results and the paper.  I will also probably need to
test everything on a two-dimensional problem.  We know that Hartmann flow
doesn't work great.  I'll have to think of another one.

\labday{Tuesday, 19 July, 2016}

\experiment{Combustion Model Inadequacy}
Nearing the end of the calibration of the reduced model.  The hold-up has been invalidating the
reduced model.  The latest results indicate that we may be able to call the calibrated
reduced model inadequate.

\subexperiment{Chemical Kinetics Inadequacy}
It would be nice to invalidate the calibrated reduced model based solely on the 0D reactor.  This has proven difficult
for the five reaction reduced model (see discussion below).  I have tried to invalidate the model with a four
reaction reduced model by removing the three-body reaction.  The algorithmic peformance is much improved (rejection
rates of around $79\%$) and chains that appear stationary.  Figures~\ref{fig:chain_1}-\ref{fig:chain_4} present
the chains from each of the four reactions.
\begin{figure}[h!]
  \centering
  \includegraphics[width=0.7\textwidth]{/h1/dsondak/Research-Notebook/figures/2016/July/reaction0_chain.pdf}
  \caption{Chains from the calibrated Arrhenius parameters for reaction 1.}
  \label{fig:chain_1}
\end{figure}
\begin{figure}[h!]
  \centering
  \includegraphics[width=0.7\textwidth]{/h1/dsondak/Research-Notebook/figures/2016/July/reaction1_chain.pdf}
  \caption{Chains from the calibrated Arrhenius parameters for reaction 2.  Note that the modified Arrhenius
           parameter, $b$, was frozen in the simulations even though we present a chain.}
  \label{fig:chain_2}
\end{figure}
\begin{figure}[h!]
  \centering
  \includegraphics[width=0.7\textwidth]{/h1/dsondak/Research-Notebook/figures/2016/July/reaction2_chain.pdf}
  \caption{Chains from the calibrated Arrhenius parameters for reaction 3.}
  \label{fig:chain_3}
\end{figure}
\begin{figure}[h!]
  \centering
  \includegraphics[width=0.7\textwidth]{/h1/dsondak/Research-Notebook/figures/2016/July/reaction3_chain.pdf}
  \caption{Chains from the calibrated Arrhenius parameters for reaction 4.}
  \label{fig:chain_4}
\end{figure}
We also provide the posterior and prior distributions for each parameter in Figures~\ref{fig:dist_1}-\ref{fig:dist_4}.  
Note that the modified Arrhenius parameter, $b$, was not actually calibrated.  It was frozen during the simulation.
\begin{figure}[h!]
  \centering
  \includegraphics[width=0.7\textwidth]{/h1/dsondak/Research-Notebook/figures/2016/July/reaction0_dist.pdf}
  \caption{Posterior and prior distributions from the calibrated Arrhenius parameters for reaction 1.}
  \label{fig:dist_1}
\end{figure}
\begin{figure}[h!]
  \centering
  \includegraphics[width=0.7\textwidth]{/h1/dsondak/Research-Notebook/figures/2016/July/reaction1_dist.pdf}
  \caption{Posterior and prior distributions from the calibrated Arrhenius parameters for reaction 2.  Note that the modified Arrhenius
           parameter, $b$, was frozen in the simulations even though we present a chain.}
  \label{fig:dist_2}
\end{figure}
\begin{figure}[h!]
  \centering
  \includegraphics[width=0.7\textwidth]{/h1/dsondak/Research-Notebook/figures/2016/July/reaction2_dist.pdf}
  \caption{Posterior and prior distributions from the calibrated Arrhenius parameters for reaction 3.}
  \label{fig:dist_3}
\end{figure}
\begin{figure}[h!]
  \centering
  \includegraphics[width=0.7\textwidth]{/h1/dsondak/Research-Notebook/figures/2016/July/reaction3_dist.pdf}
  \caption{Posterior and prior distributions from the calibrated Arrhenius parameters for reaction 4.}
  \label{fig:dist_4}
\end{figure}
It appears that we are learning quite a bit about a few parameters.  The maximum likelihood and MAP point
were obtained from the chains and the forward problem was run at each of these parameter sets.  The 
problem was also run at the nominal parameter set.  Results were compared with the detailed model and are
shown in Figure~\ref{fig:four_rxn_inad}.
\begin{figure}[h!]
  \centering
  \includegraphics[width=0.7\textwidth]{/h1/dsondak/Research-Notebook/figures/2016/July/Toft_four_rxn_inad.pdf}
  \caption{Time evolution of the temperature.  Comparisons are made between the four reaction reduced model
           for three different parameter sets and the detailed model.  Results are shown at $\phi=1$ which
           is the equivalence ratio at which the data was calibrated.}
  \label{fig:four_rxn_inad}
\end{figure}
Even after calibration, the four reaction model is inadequate.  This may provide a simple route to helping
Rebecca get her stuff going again.  I'll need to calibrate the stochastic operator on this problem and then
see how things look.  For now, I need to get the stochastic operator up and running.

\subexperiment{Diffusion Flame}
The diffusion flame will eventually be used to build up a flamelet library.  For now, I'm just using it to
see if it helps us invalidate the five reaction reduced mechanism.  Figures~\ref{fig:T_Z}-\ref{fig:H2_x} were generated
from samples of the five reaction reduced mechanism.
\begin{figure}[h!]
  \centering
  \includegraphics[width=0.7\textwidth]{/h1/dsondak/Research-Notebook/figures/2016/July/T_Z.pdf}
  \caption{Temperature as a function of mixture fraction.}
  \label{fig:T_Z}
\end{figure}
\begin{figure}[h!]
  \centering
  \includegraphics[width=0.7\textwidth]{/h1/dsondak/Research-Notebook/figures/2016/July/T_x.pdf}
  \caption{Temperature as a function of space.}
  \label{fig:T_x}
\end{figure}
\begin{figure}[h!]
  \centering
  \includegraphics[width=0.7\textwidth]{/h1/dsondak/Research-Notebook/figures/2016/July/H_Z.pdf}
  \caption{Hydrogen as a function of mixture fraction.}
  \label{fig:H2_Z}
\end{figure}
\begin{figure}[h!]
  \centering
  \includegraphics[width=0.7\textwidth]{/h1/dsondak/Research-Notebook/figures/2016/July/H_x.pdf}
  \caption{Hydrogen as a function of space.}
  \label{fig:H2_x}
\end{figure}
The plots of \ce{H} are clearly inadequate by almost any measure.  The plot of temperature with
mixture fraction also appears inadequate.  However, when plotting temperature with space, the results
look pretty good.  Other results indicate that some species are shifted in the domain.  That is, 
the flame location is not correct.  The concentration of \ce{HO2} is horrendously off to the point
where the reduced model predicts non-negligible concentrations while the detailed model predicts 
that \ce{HO2} is a trace species.  A few species, such as \ce{H2O}, \ce{H2} and \ce{O2} are more
or less correct as expected.  To quantify this a bit more, I will run a few samples from the chain
through the diffusion flame to see if I can put some error bars on the plots.  It will be difficult
to do this for the plots with mixture fraction since the mixture fraction will also have error bars.
However, maybe the spatial plots will show some inadequacy.

I propagated $1000$ samples through the diffusion flame and computed the mean and standard deviation
of the temperature at each grid point.  The results are plotted in Figure~\ref{fig:inad_T_x}.
\begin{figure}[h!]
  \centering
  \includegraphics[width=0.7\textwidth]{/h1/dsondak/Research-Notebook/figures/2016/July/inad_T_x.pdf}
  \caption{Inadequacy in temperature profile across the diffusion flame.  Dark shading represents two
           standard deviations while light shading represents one standard deviation.  Most of the
           inadequacy occurs near the boundary of the flame.}
  \label{fig:inad_T_x}
\end{figure}
The calibrated reduced model does quite well at predicting the diffusion flame.  The main inadequacies 
are found near the boundaries of the flame.  Otherwise, the detailed model falls easily within one
standard deviation of the mean.



\end{document}

%%----------------------------------------------------------------------------------------
%	LAB BOOK CONTENTS
%----------------------------------------------------------------------------------------

\labday{Thursday, 18 August, 2016}

%-----------------------------------------

\experiment{0D Reactor}
A few notes on the energy equation for the 0D reactor.  After ignoring all spatial
dependence the energy equation becomes,
\begin{align}
  \rho C_{p}\odeone{T}{t} = \dot{\omega}_{T}^{\prime} + \dot{\mathcal{Q}}.
\end{align}
Substituting for the heat release due to combustion we have
\begin{align}
  \rho C_{p}\odeone{T}{t} = -\sum_{k=1}^{N}h_{k}\lr{T}\dot{\omega}_{k} + \dot{\mathcal{Q}}.
\end{align}
Note that all quantities are on a mass basis.  The 0D reactor uses molar quantities.
\texttt{Antioch} returns mass quantities.  We need to check the units to make sure 
everything works out.

The species equation becomes
\begin{align}
  \odeone{\rho Y_{k}}{t} = \dot{\omega}_{k}.
\end{align}
Note that
\begin{align}
  \left[X_{k}\right] = \rho \frac{Y_{k}}{W_{k}} \Rightarrow \rho Y_{K} = W_{K}\left[X_{k}\right].
\end{align}
Now we have
\begin{align}
  \dot{x}_{k} = \odeone{\left[X_{k}\right]}{t} = \frac{\dot{\omega}_{k}}{W_{k}}.
\end{align}
And so the energy equation becomes
\begin{align}
  \rho C_{p}\odeone{T}{t} = - \sum_{k=1}^{N}h_{k}\lr{T}W_{k}\dot{x}_{k} + \dot{\mathcal{Q}}.
\end{align}
The specific heat is
\begin{align}
  C_{p} &= \sum_{k=1}^{N}C_{pk}Y_{k} \\
        &= \sum_{k=1}^{N}C_{pk}\frac{W_{k}}{\rho}x_{k}.
\end{align}
Then
\begin{align}
  \odeone{T}{t} = \frac{\displaystyle -\sum_{k=1}^{N}h_{k}\lr{T}W_{k}\dot{x}_{k} + \dot{\mathcal{Q}}}{\displaystyle \sum_{k=1}^{N}C_{pk}W_{k}x_{k}}.
\end{align}

%::::::::::::::::::::::::::::::::::::::::::::::::::::::::
\labday{Friday, 19 August, 2016}
%::::::::::::::::::::::::::::::::::::::::::::::::::::::::

\experiment{StochasticOP}
Trying to calibrate the stochastic operator for chemical kinetics.
Having the same problems as before with the reduced model.  Namely,
the rejection rates are simply too high and the log likelihoods are
too small.

I'm trying a few things:
\begin{itemize}
  \item Turn off stochastic operator to try to recover calibration from
        reduced model.
  \item Modify global temperature dependence to try to get better
        behavior.
  \item Played a bit with the algorithm parameters (adaptation interval
        length, etc).
\end{itemize}

Waiting to see the chains.  The global temperature dependence enters the
operator via a prefactor as
\begin{align}
  \odeone{\mathbf{x}}{t} = \mathbf{f}\lr{\mathbf{x}} + 
      g\lr{T}\left[\mathcal{S}\mathbf{x} + \mathcal{A}\lr{\mathbf{x}}\right].
\end{align}
The form that I have selected for the prefactor is
\begin{align}
  g\lr{T} = \frac{1}{2}\exp\lr{-\frac{T_{ag}}{T}}
    \left[\tanh\lr{\frac{T - T_{gi}}{\delta}} - \tanh\lr{\frac{T - T_{adi}}{\delta}}\right].
\end{align}
This introduces three new parameters to calibrate:
\begin{enumerate}
  \item $T_{ag}$, the global activation energy;
  \item $T_{gi}$, global ignition temperature;
  \item $T_{adi}$, global adiabatic temperature.
\end{enumerate}
\begin{figure}[ht!]
  \includegraphics[width=\textwidth]{August/global_T.pdf}
  \caption{Prefactor for the stochastic operator.}
  \label{fig:gofT}
\end{figure}
Figure~\ref{fig:gofT} shows the behavior of the prefactor.  The hope is that this
allows the stochastic operator to turn on and off as needed.  The Arrhenius form
basically just sets the amplitude of the prefactor and the premultiplication of 
$1/2$ just scales the hyperbolic tangents to unity (no real reason to do this
other than it helps humans see what's going on).  The smoothing factor
$\delta$ is selected to be unity.

%----------------------------------------------------------------------------------------
%	LAB BOOK CONTENTS
%----------------------------------------------------------------------------------------

%::::::::::::::::::::::::::::::::::::::::::::::::::::::::
\labday{Friday, 2 September, 2016}
%::::::::::::::::::::::::::::::::::::::::::::::::::::::::

\experiment{StochasticOP}
We have decided to only calibrate the inadequacy parameters (and hyperparameters)
and to freeze the model parameters.  The reason behind this is because inference
of the model parameters is particularly difficult as evidenced by the high
rejection rates.  Our current approach is defensible in the near-term because we
are really only interested in calibration of the inadequacy parameters.  We are
accepting the model parameters as truth (and truth be told, they are pretty 
darn good).

I have calibrated three versions of the inadequacy model: 1.) Using no global
temperature dependence; 2.) using a global temperature dependence in Arrhenius
form and 3.) using a global temerature dependence in the hyperbolic tangent form.
With no global temperature dependence (and no model parameter inference) the 
calibration is remarkably quick for the first $30000$ iterations.  However, 
the reaction rates for the catchall reactions eventually cause the system to 
become quite stiff and at iteration $\sim 34000$ the model evaluation times 
are on the order of tens of minutes.  The rejection rates are still rather ``low''
by our standards.  They hover somewhere between $76\%$ and $85\%$.  However,
the likelihoods are very small (essentially zero).  For example, the loglikelihood
at iteration $36000$ is around $-25000$.

Using the Arrehnius global temperature dependence, the model evaluations are
a bit slower and the rejecton rates are considerably higher ($\sim 96\%$).  
The likelihoods are larger but still essentially zero ($-22000$ at iteration
$36000$).

With the global hyperbolic temperature dependence, the model evaluations 
are comparable to those from the global Arrhenius temperature dependence. 
The rejection rates are $\sim 91\%$ and the likelihoods are larger still 
(but still essentially zero, e.g. $-7500$ at iteration $38000$).

Note that the calibrations described above were run using data from 
ten calibration points as opposed to data from twenty points as I had been
doing previously.  The main point is that the calibration data should be 
a reasonable representation of curves.  If ten points gives a decent 
representation then adding more points causes more work and does not 
give much in return.

Additionally, the log likelihood actually has an additional factor that 
I am not reporting.  Recall, the likelihood that we are using is 
\begin{align}
  L = \frac{1}{\lr{2\pi}^{n_{d}/2}\left|\Sigma\right|^{1/2}}
     \exp\lr{-\frac{1}{2}\lr{\mathbf{d} - \mathbf{x}}^{T}\Sigma^{-1}\lr{\mathbf{d} - \mathbf{x}}}
\end{align}
where $\Sigma$ is a diagonal matrix of variances, $n_{d}$ is the number of data points (here
$n_{d} = 10$, $\mathbf{d}$ is the truth data and $\mathbf{x}$ is the model data.  Note that for a 
diagonal matrix 
\begin{align}
  \left|\Sigma\right| = \prod_{i=1}^{n}\sigma_{i}
\end{align}
where $n$ is the size of the matrix.  Here $\sigma_{i} = 5\cdot 10^{-3}$ for $i=1,\ldots 7$
(for the first seven species), $\sigma_{8} = 2\cdot 10^{3}$ for the temperature 
and $\sigma_{9} = 10$ for the ignition temperature data.  Hence
\begin{align}
  \left|\Sigma\right| = \lr{5\cdot 10^{-3}}^{7} \cdot 2\times 10^{3} \cdot 10.
\end{align}
The log-likelihood is 
\begin{align}
  \log\lr{L} = -\log\lr{\lr{2\pi}^{n_{d}/2}\left|\Sigma\right|^{1/2}} - 
      \frac{1}{2}\lr{\mathbf{d} - \mathbf{x}}^{T}\Sigma^{-1}\lr{\mathbf{d} - \mathbf{x}}.
\end{align}
With our values, the first term is $4.403$.  It has very little effect on the full
log likelihood.

\experiment{Combustion Model Inadequacy}
Trying to propagate the fully stochastic operator model through the flamelet model to generate 
the uncertain flamelet.

\subexperiment{Diffusion Flame}
Considerable changes had to be made to the Cantera input files to propagate the stochastic 
operator through the diffusion flame.  Rebecca's original template was incomplete.  First of 
all, the NASA polynomials of order $n$ must have $n$ coefficients specified.  Unspecified 
coefficients are not assumed to be zero and an error is thrown.  The NASA polynomial 
coefficients do not map directly to the quadratic form assumed by Rebecca for the inference. 
Recall that the NASA polynomial for the enthalpy is given by 
\begin{align}
  \frac{h}{RT} = a_{0} + \frac{a_{1}}{2}T + \frac{a_{2}}{3}T^{2} + \frac{a_{3}}{4}T^{3} + 
       \frac{a_{4}}{5}T^{4} + \frac{a_{5}}{T}
\end{align}
whereas Rebecca's parameterization for the catchall species looks like 
\begin{align}
  h^{\prime} = \alpha_{0} + \alpha_{1}T + \alpha_{2}T^{2}.
\end{align}
Thus, to insure consistency in the input file, we have the relationships,
\begin{align}
  a_{0} = \frac{\alpha_{1}}{R}, \quad a_{1} = \frac{2\alpha_{2}}{R}, \quad a_{5} = \frac{\alpha_{0}}{R}
\end{align}
and $a_{i} = 0$ for $i=2, 3, 4$.

Another difficulty is how exactly to specify the catchall species.  We know that $\ce{H}^{\prime}$ 
is made up of hydrogen.  However, Cantera gets confused between two species which has the exact 
same compositions.  We initially tried to define our own atom (a dummy atom) which would have
the exact same properties as $\ce{H}$.  However, this resulted in Cantera errors.  Both
approaches caused Cantera to fail in the \texttt{set\_initial\_guess} method at the \texttt{equilibrate} 
step.  In that step, Cantera attempts to set a reasonable initial guess to kick off the simulation 
based off an equilibrium solution. 

The way that I got around these convergence issues was to change the \texttt{set\_initial\_guess} 
method in addition to the actual initial guesses (this is actually independent somehow).  
The \texttt{set\_initial\_guess} method must be called one way or another.  It appears to 
set up important solution objects.  To get the inadequacy model to run, I used a previous 
solution from the reduced model.  I also modified the \texttt{set\_initial\_guess} method 
to accept a flag called \texttt{calc\_eq}.  When this flag is set to \texttt{False} the 
\texttt{equilibrate} routine is not called and no initial guess is returned.  The 
\texttt{set\_initial\_guess} method still sets up the necessary solution objects for the 
simulation.  In our application, when a previous solution is passed to the diffusion flame
module, the flag \texttt{calc\_eq} is automatically set to \texttt{False}.  After 
\texttt{set\_initial\_guess} is called, the solution fields are populated based on the 
previous solution and the catchall fields are set to zero.  The inadequacy model 
simulation is then able to run to completion modulo time integration convergence 
problems (see below). 

An additional modification that had to be made was to add a global Arrhenius 
prefactor to the inadequacy model.  This was not a problem.  The activiation 
energy field was specified in the reactions corresponding to the inadequacy 
model.  Note that the calibration was performed for the global activation 
temperature which is related to the global activation energy simply through 
the universal gas constant via $T_{ag} = E_{ag} / R$. 

The time integration failed to converge in the absence of the global 
temperature dependence.  Using the global temperature dependence, the time 
integration was successful but I still don't know why it didn't converge 
originally.  Remember, the 0D reactor time integrator converges even in 
the absence of the global temperature dependence (although it does struggle 
once in a while).

In any case, we can look at solutions from the diffusion flame and compare 
the detailed, reduced and inadequacy models.  Note that the chains have 
still not been completely propagated through the diffusion flame so there 
may be some parameter values that are really tough to deal with.

Figure~\ref{fig:T_diffusion_flame} presents temperature profiles for the 
diffusion flame.  For this parameter set, the reduced model looks better 
than the inadequacy model.  The inset in~\ref{fig:T_diffusion_flame} indicates 
that the inadequacy model is active near the boundary of the flame before 
anything should happen.  However, the temperature change is only about one 
degree Kelvin which is quite small.
\begin{figure}[ht!]
  \includegraphics[width=\textwidth]{September/Temperatures.pdf}
  \caption{Diffusion flame emperature profiles from the detailed,
           reduced, and inadequacy models.}
  \label{fig:T_diffusion_flame}
\end{figure}

The catchall species are hopefully only active in the flame. 
Figure~\ref{fig:catchall_profiles} helps to confirm this.  We observe that 
the catchall species mass fractions only turn on inside the flame region and 
appear to be inactive outside the flame.
\begin{figure}[ht!]
  \includegraphics[width=\textwidth]{September/Catchalls.pdf}
  \caption{Profiles of the catchall mass fractions in the diffusion flame.}
  \label{fig:catchall_profiles}
\end{figure}

We can consider the heat released due to combustion and the species reaction 
rates in order to gain a more fine-grained understanding of the behavior 
of the inadequacy model.  Figure~\ref{fig:heat_release} shows the heat of 
combustion from each model.  The models are all in pretty good agreement 
except for the region just before the flame on the fuel side of the domain. 
In that region, the inadequacy model first releases heat and then absorbs 
heat before finally releasing heat in agreement with the other two models.
\begin{figure}[ht!]
  \includegraphics[width=\textwidth]{September/heat_release.pdf}
  \caption{Heat released in the diffusion flame for the detailed, 
           reduced, and inadequacy models.}
  \label{fig:heat_release}
\end{figure}

Figures~\ref{fig:reaction_rates_detailed}-~\ref{fig:reaction_rates_inadequacy} 
present species reaction rates for the detailed, reduced, and inadequacy 
models repectively.  This gives us a sense of which species are active or 
inactive in specific regions of the domain.  Interestingly, the catchall 
reaction rates do not appear to be active outside of the flame.  However, 
the inadequacy solution is biases towards the fuel side of the flame.
\begin{figure}[ht!]
  \includegraphics[width=\textwidth]{September/reaction_rates_detailed.pdf}
  \caption{Reaction rates for each species in the diffusion flame for the
           detailed model.  The temperature profile from the detailed 
           model is included as a reference for flame location.}
  \label{fig:reaction_rates_detailed}
\end{figure}

\begin{figure}[ht!]
  \includegraphics[width=\textwidth]{September/reaction_rates_reduced.pdf}
  \caption{Reaction rates for each species in the diffusion flame for the
           reduced model.  The temperature profile from the detailed 
           model is included as a reference for flame location.}
  \label{fig:reaction_rates_reduced}
\end{figure}

\begin{figure}[ht!]
  \includegraphics[width=\textwidth]{September/reaction_rates_inadequacy.pdf}
  \caption{Reaction rates for each species in the diffusion flame for the
           inadequacy model.  The temperature profile from the detailed 
           model is included as a reference for flame location.}
  \label{fig:reaction_rates_inadequacy}
\end{figure}

Some possible next steps are included in the following:
\begin{itemize}
  \item Plot the enthalpies for each species.
  \item Plot MAP point solutions.
  \item Plot maximum likelihood solutions.
  \item Start propagation of chain through the flamelet code
        to generate the uncertain flamelet.
  \item Work on reformulation of inadequacy model.
  \item Try different global activation energies.
\end{itemize}
























%-----------------------------------------

\bibliographystyle{plain}
\bibliography{references}

%-----------------------------------------
%\end{addmargin}

\end{document}


